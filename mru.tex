
\chapter{Le mouvement rectiligne uniforme (MRU)}
Le \motcle{mouvement rectiligne uniforme} est celui d'un objet se déplaçant en ligne droite et à vitesse constante.

\section{Propriétés du MRU}
%je vais tenter de ne pas mettre le développement expérimental dans le cours, il sera fait au labo
Dans un MRU, la vitesse est constante. Cela implique que la position est proportionnelle au temps : pour une durée deux fois plus grande, le déplacement effectué est deux fois plus important. Nous verrons plus tard, dans le cours sur la dynamique, qu'un MRU prend place lorsque la résultante des forces agissant sur un corps est nulle.

\subsection{Graphiques types du repos et du MRU}
%je fais la manip avec les élèves pour établir les graphiques.
\begin{tabularx}{\linewidth}{m{.1\linewidth} X X}
    \hline
                          & Position en fonction du temps & Vitesse en fonction du temps \\
    \hline
    \rotatebox{90}{Repos} &                               &                              \\[4cm]
    \hline
    \rotatebox{90}{MRU}   &                               &                              \\[4cm]
    \hline
    \rotatebox{90}{MRU}   &                               &                              \\[4cm]
    \hline \hline
\end{tabularx}

\newpage

\subsection{Équations horaire du MRU}
Les équations horaires d'un mobile en mouvement sont celles qui renseignent sur la valeur d'une propriété de ce mobile, comme la position ou la vitesse, à chaque instant.

\subsubsection{Équation horaire de la position}
Dans un MRU, l'équation permettant de connaître la position à chaque instant est:
\begin{encadre}
    \begin{equation}
        x(t)=x_0+v \cdot \Delta t
    \end{equation}
    où :
    \begin{itemize}[label= \textbullet]
        \item \(x(t)\) est la position à un instant t, en \(\unit{[m]}\);
        \item \(x_0\) est la position à l'instant 0, la position de départ, en \(\unit{[m]}\);
        \item \(v\) est la vitesse, en \(\unit{[m/s]}\) ou \(\unit{[m \cdot s^{-1}]}\);
        \item \(\Delta t\) est le temps écoulé entre l'instant 0 et l'instant t, la durée du mouvement, en \([s]\).
    \end{itemize}
\end{encadre}

\subsubsection{Équation horaire de la vitesse}
L'équation permettant de connaître la vitesse à un instant donné est:
\begin{encadre}
    \begin{equation}
        v(t)=cst
    \end{equation}
    où :
    \begin{itemize}[label= \textbullet]
        \item \(v\) est la vitesse, en \(\unit{[m/s]}\) ou \(\unit{[m \cdot s^{-1}]}\).
    \end{itemize}
\end{encadre}
La vitesse étant constante, c'est une propriété fondamentale du MRU, cette équaton est très simple.

\newpage

\section{Exercices}
\begin{exercise}
    Convertir les vitesses suivantes en \unit{[m/s]} : 72 \unit{[km/h]}; 5\unit{[km/h]}; 30 km/s
\end{exercise}
\begin{solution}
    Il faut multiplier par 3.6.
\end{solution}

\begin{exercise}
    Convertir en \unit{[km/h]}: 10\unit{[m/s]} ; 330\unit{[m/s]}
\end{exercise}
\begin{solution}
    Il faut diviser par 3.6
\end{solution}


\begin{exercise}
    Un athlète court un marathon (42,195 km) en 2h5min42s. Calculer sa vitesse moyenne.
\end{exercise}
\begin{solution}
\end{solution}


\begin{exercise}
    Je pars de la maison à 8h20min30s. Le compteur de la voiture indique 437,2 km.
    Je me gare près du bureau à 9h2min40s. Le compteur indique 486,5km.
    Calculer la vitesse moyenne durant le trajet (\unit{[m/s]} et\unit{[km/h]}).
\end{exercise}
\begin{solution}
    \(v=19,486\unit{[m/s]}\)
\end{solution}


\begin{exercise}
    Lors d'une épreuve contre la montre de 20km, un cycliste parcourt les 10 premiers km à 40\unit{[km/h]}de moyenne. Les 10 derniers sont en côte et il les franchit à 20\unit{[km/h]}de moyenne. Quelle est sa vitesse moyenne sur l'ensemble de l'épreuve ?
\end{exercise}
\begin{solution}
    \(v=7,407\unit{[m/s]}\)
\end{solution}


\begin{exercise}
    Lors de l'Ironman de Malaisie du 23 février 2003, le vainqueur, Luc Van Lierde, a d'abord nagé les 3,8 km en 47m54s. Il a ensuite mis 53s pour se changer une première fois, enfourché sa bicyclette et roulé pendant 4h41min02s pour parcourir les 180,2km. Finalement, après s'être changé en 1min55s, il a couru les 42,2 derniers km de l'épreuve en 2h59min33s. Calculer sa vitesse moyenne pour chacune des épreuves et pour l'ensemble du triathlon.
\end{exercise}
\begin{solution}
\end{solution}


\begin{exercise}
    Un homme marche en ligne droite jusqu'au coin de la rue pour poster une lettre. Là, il rencontre un ami, bavarde quelques instants puis revient chez lui en courant toujours en ligne droite.
    Tracer l'allure des graphiques x(t) et v(t).
\end{exercise}
\begin{solution}
\end{solution}

\begin{exercise}
    Un ascenseur de puits de mine situé à 1500 m de profondeur remonte à la surface. Le mouvement de remontée est un mouvement rectiligne uniforme. La vitesse de translation est de 125 m/min.
    \begin{enumerate}[label=\alph*)]
        \item Déterminer l'équation horaire de position de l'ascenseur.
        \item Déterminer la durée totale de la remontée.
    \end{enumerate}
\end{exercise}
\begin{solution}
    \begin{enumerate}[label=\alph*)]
        \item \(x(t)=-1500 + 2,083 \cdot \Delta t\)
        \item \(\Delta t=720[s]\)
    \end{enumerate}
\end{solution}

\begin{exercise}
    Une automobile roule à vitesse constante de 60 \unit{[km/h]}sur une autoroute quand une autre automobile la dépasse à la vitesse maximale permise de 100\unit{[km/h]}. Quelle distance sépare les deux automobiles 5 s plus tard?
\end{exercise}
\begin{solution}
    \(\Delta x=55,56\unit{[m]}\)
\end{solution}

\begin{exercise}
    Le graphique ci-contre représente les cinq étapes (A à E) du voyage d'un cycliste. Durant quelle(s) étape(s) :
    \begin{enumerate}[label=\alph*)]
        \item Sa vitesse est-elle positive ?
        \item Sa vitesse est-elle nulle ?
        \item Sa vitesse est-elle négative ?
        \item Sa vitesse a-t-elle la plus grande valeur positive ?
        \item Le cycliste roule-t-il le plus vite ?
        \item La plus grande distance est-elle parcourue ?
    \end{enumerate}
    \begin{tikzpicture}
        \tkzInit[xstep=1,ystep=1,xmax=7,ymax=6]
        \tkzGrid
        \tkzDrawX[label={$Temps[s]$},below left=25pt]
        \tkzDrawY[label={$Position\unit{[m]}$},right=5pt]
        \tkzAxeXY[label={}] %This macro combines the four macros: \tkzDrawX\tkzDrawY \tkzLabelX\tkzLabelY
        \tkzDefPoints{0/0/O, 1/2/A, 3/5/B, 4/5/C, 5/1/D, 6/0/E};
        \tkzDrawSegments[color=black](O,A A,B B,C C,D D,E);
        \tkzDrawPoints(O,A,B,C,D,E);
        \tkzLabelPoints[above](O,A,B,C,D,E);
    \end{tikzpicture}

\end{exercise}
\begin{solution}
\end{solution}

\begin{exercise}
    Deux voitures partent en même temps de deux villes A et B distantes de 120km. Elles roulent l'une vers l'autre. La voiture partie de A roule à 60km/h, celle partie de B à 90km/h. Déterminer graphiquement et par calcul l'heure et la distance du croisement.
\end{exercise}
\begin{solution}
\end{solution}

\begin{exercise}
    Deux automobiles A et B partent d'un même endroit sur la même route rectiligne. Elles roulent dans le même sens. A part à 13h et B à 13h30. A roule à 80\unit{[km/h]}et B à 110km/h. Déterminer l'heure et l'endroit de la rencontre.
\end{exercise}
\begin{solution}
\end{solution}

\begin{exercise}
    Le graphique ci-dessous décrit le mouvement d'une voiture. Tracez le graphique v(t) correspondant.

    \begin{tikzpicture}
        \tkzInit[xstep=0.5,ystep=10,xmax=1,ymax=50]
        \tkzGrid
        \tkzDrawX[label={$Temps[heure]$},below left=25pt]
        \tkzDrawY[label={$Position[km]$},right=5pt]
        \tkzAxeXY[label={}] %This macro combines the four macros: \tkzDrawX\tkzDrawY \tkzLabelX\tkzLabelY
        \tkzDefPoints{0/0/O, 0.5/20/A, 1/50/B};
        \tkzDrawSegments[color=black](O,A A,B);
        \tkzDrawPoints(O,A,B);
        %\tkzLabelPoints[above](O,A,B,C,D,E);
    \end{tikzpicture}
\end{exercise}

\begin{exercise}
    Deux voitures, une verte et une rouge, roulent sur une même route (voir graphique).
    \begin{enumerate}[label=\alph*)]
        \item Quelle est la voiture la plus rapide ?
        \item Que se passe-t-il à l'instant t1 ?
    \end{enumerate}
    \begin{tikzpicture}
        \tkzInit[xstep=1,ystep=1,xmax=6,ymax=4]
        \tkzGrid
        \tkzDrawX[label={$Temps[s]$},below left=25pt]
        \tkzDrawY[label={$Position\unit{[m]}$},right=5pt]
        \tkzAxeXY[label={}] %This macro combines the four macros: \tkzDrawX\tkzDrawY \tkzLabelX\tkzLabelY
        \tkzDefPoints{0/0/O, 6/3/A, 2/0/B, 6/4/C, 4/2/D};
        \tkzDrawSegment[color=red](O,A);
        \tkzDrawSegment[color=green](B,C);
        \tkzDrawPoints(D);
        \tkzLabelPoint[above](D){\(t_1\)};
    \end{tikzpicture}

\end{exercise}
\begin{solution}
    La voiture verte est plus rapide. À l'instant t1, la voiture verte dépasse la rouge.
\end{solution}

\newpage

\begin{exercise}
    Le graphique horaire d'une voiture en MRU est donné ci-dessous. En examinant soigneusement ce graphique :
    \begin{enumerate}[label=\alph*)]
        \item Donner la position initiale de la voiture
        \item Calculer sa vitesse et vérifier qu'elle est constante.
        \item Ecrire les équations horaires correspondant à ce mouvement (position, vitesse et accélération).
        \item Calculer sa position après 2 minutes.
    \end{enumerate}
    \begin{tikzpicture}
        \tkzInit[xstep=1,ystep=50,xmax=7,ymin=-100,ymax=150]
        \tkzGrid
        \tkzDrawX[label={$Temps[s]$},below left=25pt]
        \tkzDrawY[label={$Position\unit{[m]}$},right=5pt]
        \tkzAxeXY[label={}] %This macro combines the four macros: \tkzDrawX\tkzDrawY \tkzLabelX\tkzLabelY
        \tkzDefPoints{0/-75/O, 7/100/A};
        \tkzDrawSegment[color=black](O,A);
    \end{tikzpicture}
\end{exercise}
\begin{solution}
\end{solution}

\begin{exercise}
    II y a 200 km entre Arlon et Bruxelles. A 9 h, un autocar quitte Bruxelles vers Arlon. A 10 h, une voiture  part d'Arlon en direction de Bruxelles. La vitesse du car est de 90\unit{[km/h]}, celle de la voiture de 120\unit{[km/h]}. Quand et où se croisent-ils ?
\end{exercise}
\begin{solution}
\end{solution}

\begin{exercise}
    Laurent quitte la maison à 8h et marche à 4 \unit{[km/h]}pour se rendre à l'école. A 8 h 15 min, son grand-père se rend compte qu'il a oublié son journal de classe et enfourche son vélo pour le lui apporter en roulant à 15\unit{[km/h]}. Trouvez où et quand il rattrapera son petit-fils distrait. Il y a 2 km entre la maison et l'école.
\end{exercise}
\begin{solution}
\end{solution}

\begin{exercise}
    Deux villes A et B sont séparées par une distance de 100 km. A 8 h, un cycliste quitte A à la vitesse constante de 20\unit{[km/h]}. A 8h40, un motocycliste quitte B et se dirige vers A à la vitesse constante de 45\unit{[km/h]}.
    \begin{enumerate}[label=\alph*)]
        \item Où et quand vont-ils se rencontrer ?
        \item Quand seront-ils séparés par une distance de 45 km ?
        \item Quelle vitesse aurait dû avoir la moto pour croiser le cycliste à 10h30 ?
        \item Quelle est la durée séparant leur passage en une ville située à 30 km de A ?
    \end{enumerate}
\end{exercise}

\begin{exercise}
    Deux véhicules partent de A à 10h et se dirigent vers B distante de A de 130km. La première voiture se déplace à la vitesse constante de 80\unit{[km/h]}. La seconde part à la vitesse constante de 60\unit{[km/h]}. Après une demi-heure, elle s'arrête pendant 10 min puis repart à la vitesse constante de 120km/h.
    \begin{enumerate}[label=\alph*)]
        \item Où et quand les 2 voitures vont-elles se dépasser ?
        \item Quelle voiture va arriver la première ? Calculer son avance en minutes.
    \end{enumerate}
\end{exercise}

\begin{exercise}
    A l'instant t = 0s, un coureur 1 part d'un point A et court à la vitesse constante de 5\unit{[m/s]}. Au même instant, un coureur 2 part d'un point B, situé 100m devant A et court à la vitesse de 2,5\unit{[m/s]}.
    Au bout de combien de temps et à quelle distance de l'origine, le coureur 1 rattrape-t-il le coureur 2 ? (Résolution algébrique uniquement)
\end{exercise}

\begin{exercise}
    Deux piétons A et B se déplacent dans le même sens sur une route rectiligne. La vitesse de A est 5,4\unit{[km/h]}, celle de B est 3,6\unit{[km/h]}. La distance qui les sépare à t = 0 est 80 m, B étant en avance sur A.
    \begin{enumerate}[label=\alph*)]
        \item À quelle date t A dépassera-t-il B ?
        \item Quelle sera alors la distance parcourue par chaque piéton depuis l'instant t = 0 ?
    \end{enumerate}
\end{exercise}

\begin{exercise}
    Robin des Bois aperçoit Marianne qui a faussé compagnie au shérif de Nottingham et qui s'éloigne en courant dans la forêt à une vitesse constante de 9\unit{[km/h]}. Au moment où il l'aperçoit, la belle a 50m d'avance sur lui. On suppose que Robin des Bois avance à vitesse constante de 18 \unit{[km/h]}sur son cheval au moment où il aperçoit Marianne. Au bout de combien de temps rejoindra-t-il Marianne ?
\end{exercise}

\begin{exercise}
    Un train quitte Bruxelles en direction de Paris à 14h05min à la vitesse moyenne de 240km/h. Un autre train quitte Paris en direction de Bruxelles à 14h15min avec une vitesse moyenne de 280km/h. La distance entre Bruxelles et Paris est de 300km. A quelle distance de Bruxelles ces trains se croisent-ils et à quelle heure (heure, minutes et secondes)?
\end{exercise}

\begin{exercise}
    Un voleur part en courant à la vitesse constante de 24km/h, un policier parti 40s plus tard cherche à le rejoindre en courant à la vitesse moyenne de 28km/h.
    \begin{enumerate}[label=\alph*)]
        \item Le voleur pourra-t-il être arrêté à temps si un complice se trouve 400m plus loin avec une voiture puissante ?
        \item Calculer la durée de la poursuite et la distance parcourue
    \end{enumerate}
\end{exercise}

\begin{exercise}
    Un lièvre et une tortue font la course. Ils courent tous-deux à une vitesse uniforme: 1 \unit{[m/s]} pour le lièvre et 0,2 \unit{[m/s]} pour la tortue. La course a lieu sur un parcours rectiligne de 150 mètres. La tortue reçoit un avantage au départ: elle commence sa course à 120 mètres de la ligne de départ.
    \begin{enumerate}[label=\alph*)]
        \item Quelle est la position du lièvre après 60 secondes de course ? Où se trouve la tortue au même moment?
        \item Où se trouve le lièvre quand la tortue a parcouru 20 mètres?
        \item Qui gagne la course?
    \end{enumerate}
\end{exercise}
