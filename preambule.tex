%importation des packages
\documentclass[11pt,twoside,DIV=calc]{scrbook}
\usepackage{scrlayer-scrpage} %équivalent de fancyhdr pour kommascript
\usepackage[french]{babel}
\usepackage[babel=true]{csquotes} % csquotes va utiliser la langue définie dans babel
\usepackage[shortlabels]{enumitem} %permet de faire \begin{enumerate}[a)]
\usepackage[dvipsnames]{xcolor}
\usepackage{tabularx}
\usepackage{amsmath} %formules mathématiques
\usepackage{amssymb} %symboles mathématiques
\usepackage{amsfonts} %polices mathématiques
\usepackage{graphicx}
\usepackage{multido}%pour la macro pointillés
\usepackage{qrcode}
%\usepackage{svg}
%\svgpath{{./images}} % <- using \svgpath to avoid warning
\usepackage{tcolorbox}
\tcbuselibrary{theorems}
\tcbuselibrary{skins,listings}
\usepackage{siunitx}
\sisetup{locale = FR}
\newtcbtheorem[number within=section]{mytheo}{}%
{colback=green!5,colframe=green!35!black,fonttitle=\bfseries}{th}
%\tcbset{myformula/.style={colback=yellow!10!white,colframe=red!50!black,
%      every box/.style={highlight math style={colback=LightBlue!50!white,colframe=Navy}}
%    }}
%\tcbset{colback=yellow!10!white, colframe=red!50!black, highlight math style= {enhanced,colframe=red,colback=red!10!white,boxsep=0pt}}

%hyperliens
\usepackage{hyperref}
\hypersetup{
  colorlinks,
  citecolor=black,
  filecolor=black,
  linkcolor=black,
  urlcolor=black
}


%police
%\usepackage{libertine} %la police
\usepackage[T1]{fontenc}

%graphique tikz
\usepackage{tikz}
\usepackage{tkz-base}
\usepackage{tkz-fct}
\usepackage{tkz-euclide}

%chemin vers les images
\graphicspath{ {./images/} }

%exercices xsim
\usepackage[clear-aux]{xsim}
\DeclareExerciseTagging{difficulty}
\xsimsetup{
  path=xsim,
  load-style = layouts ,
  exercise/template = runin ,
  solution/template = runin,
  difficulty={*,**,***},
}



%math
\everymath{\displaystyle} %pour que les équations soient bien présentées


%les variables du document
\subject{Physique(2h)}
\title{\Huge{Cinématique}}
\subtitle{5\textsuperscript{ème} GT}
\author {J.N. Gautier}
\date{}



%mise en page générale
\KOMAoptions{DIV=last}
\usepackage{setspace}%pour regler l'interligne
\onehalfspacing %interligne1.5
\setlength{\parindent}{0pt} %pas de retrait en début de paragraphe
\usepackage{caption}%mise en forme des légendes des figures
\addtokomafont{caption}{\scriptsize \itshape}
\addtokomafont{captionlabel}{\scriptsize \itshape}

%pieds de page utiliser scrlayer-scrpage
\setkomafont{pageheadfoot}{\small}
\lohead{} \rohead{} \cohead{}
\lofoot{Cinématique - physique(2h) - 5GT}
\rofoot{Page \thepage}
\cofoot{}


%style perso et macro
\newcommand{\motcle}[1]{
  \uppercase{\textbf{#1}}
}

\newcommand{\pointilles}[1]{
  \begin{itemize}[label={}]
    \multido{}{#1}{\item \dotfill}
  \end{itemize}
}

\newenvironment{encadre}{\begin{center}\begin{minipage}[c]{0.8\linewidth}\begin{tcolorbox}[colframe=red!50!yellow,colback=red!5!white]}{\end{tcolorbox}\end{minipage}\end{center}}



